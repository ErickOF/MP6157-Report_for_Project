\section{Conclusions}
The Kalman filter provides an optimal method for estimating the state of a dynamic system. By efficiently combining noisy sensor measurements with the system's dynamic model, it yields accurate and robust estimation of the true state. Its adeptness at mitigating measurement noise and process disturbances bolsters the accuracy and dependability of state estimation, proving invaluable in scenarios demanding meticulous measurements. It is versatile and adaptable to a wide range of dynamic systems, including linear and nonlinear systems, making it applicable across various domains such as aerospace, robotics, finance, and signal processing. With its recursive nature and computational efficiency, the Kalman filter is a prime candidate for real-time applications, facilitating continuous state estimation and system control while imposing minimal computational overhead. \\

Arduino serves as a versatile platform appealing to hobbyists, educators, and professionals alike. Offering a pliable framework, it fosters the creation of an extensive range of electronic projects. Boasting an intuitive interface and abundant online support, Arduino serves as a gateway to the realms of electronics and programming, facilitating learning and experimentation with ease. Its compatibility with a various sensors, actuators, and communication modules streamlines incorporation into diverse projects, empowering users to forge interactive and interconnected systems effortlessly. Moreover, its straightforward operation, cost-effectiveness, and swift prototyping prowess render it an ideal candidate for iterative refinement and speedy proof-of-concept validation, expediting the genesis of inventive solutions. \\

The Simulink with Arduino Add-on combines the Simulink's robust simulation capabilities with the Arduino's versatility, enabling users to prototype and test intricated control algorithms and system designs in a virtual environment prior hardware implementation. This integration facilitates smooth communication between Simulink models and Arduino hardware, allowing for real-time interaction and control of physical devices directly from the Simulink environment. By leveraging Simulink's user-friendly graphical interface and extensive block library in alongside with Arduino's simplicity and hardware flexibility, the add-on expedites the development process, facilitating swift iteration and enhancement of designs for efficient project advancement.


