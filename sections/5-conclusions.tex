\section{Conclusions}
Heating and temperature control systems play a crucial role in enhancing both energy efficiency and occupant comfort. Accurately controlling indoor temperatures is indispensable across several fields, including but not limited to biotechnology, biology, food supply chain, transportation, and automotive industries. These systems ensure optimal thermal comfort while providing a high degree of customization and flexibility to accommodate diverse user requirements and building characteristics. By minimizing energy wastage, they facilitate cost savings and contribute significantly to environmental sustainability efforts. \\

The least-squares filter exhibits remarkable robustness in handling data susceptible to random errors or fluctuations, showcasing its adaptability to diverse degrees of data complexity. While it excels in delivering accurate parameter estimations, particularly with large data, it does entail heightened computational demands as dataset size increases, necessitating careful resource allocation. Nevertheless, its versatility renders it invaluable across a wide range of fields and disciplines, underscoring its universal utility and relevance. \\

Simulink presents a robust platform for multi-domain modelling and simulation, empowering engineers and researchers to design and simulate complex systems that span multiple domains. Its intuitive graphical user interface and extensive library of pre-built blocks streamline the integration of diverse components and subsystems, facilitating accelerated prototyping and comprehensive system-level analysis and validation. With its user-friendly block diagram interface, users can quickly iterate on system designs, fine-tune parameters, and evaluate performance in real-time, thereby trimming development timelines and expenses. Furthermore, Simulink's seamless integration with MATLAB provides a cohesive environment for learning and applying computational techniques across various engineering and scientific disciplines.
